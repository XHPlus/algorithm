\documentclass{article}
\usepackage{algorithm}
\usepackage{algorithmicx}
\usepackage{algpseudocode}
\usepackage{ctex}
\usepackage{float}
\usepackage{amsmath}
\usepackage{listings}
\usepackage{xcolor}
\usepackage{geometry}
\geometry{a4paper,scale=0.8}
\lstset { %
	language=C++,
	backgroundcolor=\color{black!5}, % set backgroundcolor
	basicstyle=\footnotesize,% basic font setting
	breaklines=true,
}
\algnewcommand{\Initialize}[1]{%
	\State \textbf{Initialize:}
	\Statex \hspace*{\algorithmicindent}\parbox[t]{.8\linewidth}{\raggedright #1}
}
\begin{document}
	\section{问题描述}
	\begin{itemize}
		\item 用分支定界算法求以下问题:\\
		某公司于乙城市的销售点急需一批成品,该公司成品生产基地在甲城市。\\
		甲城市与乙城市之间共有 n 座城市,互相以公路连通。\\
		甲城市、乙 城市以及其它各城市之间的公路连通情况及每段公路的长度由矩阵 M1 给出。\\
		每段公路均由地方政府收取不同额度的养路费等费用,具体数额由矩 阵M2给出。\\
		请给出在需付养路费总额不超过 1500 的情况下,该公司货车运送其 产品从甲城市到乙城市的最短运送路线。\\
		具体数据参见文件:\\
		M1.txt: 各城市之间的公路连通情况及每段公路的长度矩阵(有向 图); 甲城市为城市 Num.1,乙城市为城市 Num.50。\\
		M2.txt: 每段公路收取的费用矩阵(非对称)。
	\end{itemize}
	
	
	\section{算法思路}
		\subsection{距离下界和费用下界}
			为了在搜索时进行尽可能的剪枝,可以首先根据$dijkstra$算法计算出没有费用限制的最短路径和最小费用,然后在遍历搜索树时利用当前节点对应路径的长度$+dijkstra$算法得到的后继序列的最短长度作为距离下界,利用当前当前节点对应路径的费用$+dijkstra$算法得到的后继序列的最小费用作为费用下界。
	\begin{enumerate}
		\item 距离下界$= length\ of\ current\_sequence+minimal\ length\ from\ current\_sequence[-1]\ to\ B$
		\item 费用下界$= cost\ of\ current\_sequence+minimal\ cost\ from\ current\_sequence[-1]\ to\ B$
	\end{enumerate}
	\subsection{分支定界法}
		在构建搜索树时可以有2种选择:二叉搜索树和多叉搜索树。下面分别介绍基于两者的分支定界求解过程。
		\subsubsection{二叉搜索树}
		\begin{algorithm}[H]
			\centering
			\caption{基于二叉搜索树的分支定界法求最短路径}
			\begin{flushleft}
				\textbf{INPUT:} 城市个数$n$, 所有城市之间的距离矩阵$distance\_matrix$和费用矩阵$cost\_matrix$\\
				\textbf{OUTPUT:} 最短路径$solution$
			\end{flushleft}
			\begin{algorithmic}[1]
				
				\Function{search}{$sequence, remove, solution$}
				\State {$last\_node=sequence[-1]$}
				\State {剪枝条件1 $\gets$ sequence中刚刚添加了新元素(或remove为空)}
				\State {剪枝条件2 $\gets last\_node$可达$B$且$sequence$长度+$shortest\_path[last\_node]$长度$ \geq$ 当前解$solution$的长度 }
				\State {剪枝条件3 $\gets$ sequence费用+$min\_cost[last\_node]$费用$ >$1500}
				\If {满足剪枝条件 $1 \wedge (2 \vee 3)$}
					\State {剪枝回溯}
				\Else
					\If {$sequence$+$shortest\_path[last\_node]$是可行解且优于当前最优解solution}
						\State {更新solution}
					\EndIf
				\EndIf
				\State {找到一个不在sequence和remove且与sequence最后一个城市连通的城市c}
				\State {$SEARCH(sequence+[c], [], solution)$}
				\State {$SEARCH(sequence, remove+[c],solution)$}
				\EndFunction \\
				\Initialize {$sequence\gets[A], remove\gets[], solution\gets[]$}
				\For {$i=0 \to n-1$}
				\State {$shortest\_path[i] \gets dijkstra$算法计算从i到B的最短路径}
				\State {$min\_cost[i] \gets dijkstra$算法计算从i到B的最小费用}
				\EndFor
				\State {$SEARCH(sequence, remove, solution)$}\\
				\Return $solution$
			\end{algorithmic}
		\end{algorithm}
	\subsubsection{多叉搜索树}
		\begin{algorithm}[H]
			\centering
			\caption{基于多叉搜索树的分支定界法求最短路径}
			\begin{flushleft}
				\textbf{INPUT:} 城市个数$n$, 所有城市之间的距离矩阵$distance\_matrix$和费用矩阵$cost\_matrix$\\
				\textbf{OUTPUT:} 最短路径$solution$
			\end{flushleft}
			\begin{algorithmic}[1]
				\Function{search}{$sequence, solution$}
				\State {$last\_node=sequence[-1]$}
				\State {剪枝条件1 $\gets last\_node$可达$B$且$ sequence$长度+$shortest\_path[last\_node]$长度$ \geq$ 当前解$solution$的长度 }
				\State {剪枝条件2 $\gets$ sequence费用+$min\_cost[last\_node]$费用$ >$1500}
				\If {满足剪枝条件 $1 \vee 2$}
				\State {剪枝回溯}
				\Else
				\If {$ sequence$+$shortest\_path[last\_node]$是可行解且优于当前最优解solution}
				\State {更新solution}
				\EndIf
				\EndIf
				\State {$cities\ \gets $所有不在$sequence$且与$sequence$最后一个城市连通的城市}
				\For {$all\ c\ \in\ cities$}
					\State {$SEARCH(sequence+[c], solution)$}
				\EndFor
				\EndFunction\\
								\Initialize {$sequence\gets[A], solution\gets []$}
				\For {$i=0 \to n-1$}
				\State {$shortest\_path[i] \gets dijkstra$算法计算从i到B的最短路径}
				\State {$min\_cost[i] \gets dijkstra$算法计算从i到B的最小费用}
				\EndFor
				\State {$SEARCH(sequence, solution)$}\\
				\Return $solution$
			\end{algorithmic}
		\end{algorithm}

	\section{代码实现和结果分析}
	\subsection{代码}
	\subsubsection{时间统计timer.h}
\begin{lstlisting}
#include <sys/time.h>

#define TI(tag)\
    struct timeval time_start_##tag, time_end_##tag;\
    do {\
        gettimeofday(&time_start_##tag, NULL);\
    } while (0)

#define TO(tag, func)\
    do {\
        gettimeofday(&time_end_##tag, NULL);\
        cout << func << " " << #tag":\t" << (time_end_##tag.tv_sec - time_start_##tag.tv_sec) * 1000.0 + (time_end_##tag.tv_usec - time_start_##tag.tv_usec) / 1000.0 << " ms" << endl;\
    } while (0)
\end{lstlisting}
	\subsubsection{二叉搜索树}
\begin{lstlisting}
/*
 * solve single source shortest path problem using branch and bound algorithm
 */

#include <iostream>
#include <fstream>
#include <string.h>
#include <vector>
#include <algorithm>
#include "timer.h"

#define MAX_NODE_NUM 50
#define INT_MAX_VALUE 9999
#define COST_UPPER_BOUND 1500

using std::ifstream;
using std::vector;
using std::find;
using std::cout;
using std::endl;
using std::reverse;

void parse_input(char *filename, int distance_matrix[][MAX_NODE_NUM])
{
    ifstream f(filename);
    for (int i = 0; i < MAX_NODE_NUM; i++) {
        for (int j = 0; j < MAX_NODE_NUM; j++) {
            f >> distance_matrix[i][j];
        }
    }
}

void print_result(int current_shortest_dist, vector<int> current_shortest_path, int cost_matrix[][MAX_NODE_NUM])
{
    cout << "shortest_distance: " << current_shortest_dist << endl;
    cout << "shortest_path: ";
    for (int i = 0; i < current_shortest_path.size(); i++) {
        if (i == current_shortest_path.size() - 1)
            cout << current_shortest_path[i]+1;
        else
            cout << current_shortest_path[i]+1 << "->";
    }
    cout << endl;
    int sum = 0;
    for (int i = 1; i < current_shortest_path.size(); i++) {
        sum += cost_matrix[current_shortest_path[i-1]][current_shortest_path[i]];
    }
    cout << "cost: " << sum << endl;
}

void dijkstra(int distance_matrix[][MAX_NODE_NUM], int source,
              int shortest_distance[], int shortest_path[])
{
    // found用于标记已经得到最短距离的目标结点
    static int found[MAX_NODE_NUM];
    // 初始化为1
    memset(found, 0, sizeof(found));

    // 找出从源点到其相邻节点的最短路径
    found[source] = 1;
    for (int i = 0; i < MAX_NODE_NUM; i++) {
        shortest_distance[i] = distance_matrix[source][i];
        shortest_path[i] = source;
    }

    int min_distance;
    int min_distance_idx;

    // 找到从源点出发到其他各结点的最短路径
    for (int i = 1; i < MAX_NODE_NUM; i++) {
        min_distance = INT_MAX_VALUE;
        // 找到未被标记的路径最短的结点
        for (int j = 0; j < MAX_NODE_NUM; j++) {
            if (found[j] == 0 && shortest_distance[j] < min_distance) {
                min_distance = shortest_distance[j];
                min_distance_idx = j;
            }
        }
        // 标记该结点为找到最短路径
        found[min_distance_idx] = 1;

        // 利用该该结点的最短路径更新其他结点的最短路径
        for (int j = 0; j < MAX_NODE_NUM; j++) {
            int dist_idx_j = distance_matrix[min_distance_idx][j]; // 从该结点到j的路径
            int new_distance = min_distance + dist_idx_j; // 从source到该结点的最短路径+从该结点到j的路径
            if (found[j] == 0 && dist_idx_j != INT_MAX_VALUE && new_distance < shortest_distance[j]) {
                shortest_distance[j] = new_distance;
                shortest_path[j] = min_distance_idx;
            }
        }
    }

}

void search(vector<int> sequence, vector<int> remove,
            int sequence_distance, int sequence_cost,
            int &current_shortest_dist, vector<int> &current_shortest_path,
            int shortest_path[][MAX_NODE_NUM],
            int shortest_distance[][MAX_NODE_NUM], int min_cost[][MAX_NODE_NUM],
            int distance_matrix[][MAX_NODE_NUM], int cost_matrix[][MAX_NODE_NUM])
{
    int seq_len = sequence.size();
    int last_node = sequence[seq_len - 1];
    int opt_distance = sequence_distance + shortest_distance[last_node][MAX_NODE_NUM-1];
    int opt_cost = sequence_cost + min_cost[last_node][MAX_NODE_NUM-1];
    // 符合剪枝条件,则回溯
    //   0. 上一步搜索时向左分支
    //   1. shortest_distance[last_node][MAX_NODE_NUM-1] > 0 && sequence_distance + shortest_distance[sequence[-1]][MAX_NODE_NUM-1] > current best solution
    //   2. sequence_cost + min_cost[sequence[-1]][MAX_NODE_NUM-1] > 1500
    //   剪枝条件: 0 且 (1或2)
    if (remove.size() == 0 && ((shortest_distance[last_node][MAX_NODE_NUM-1] > 0 &&
        opt_distance >= current_shortest_dist) || opt_cost > COST_UPPER_BOUND)) {
        // 剪枝回溯
        return;
    } else {
        // 判断是否是可行解
        if (remove.size() == 0 && shortest_distance[last_node][MAX_NODE_NUM-1] > 0) {
            // 上一步是向左分支且当前序列去往B的后继路径存在时,才可能产生可行解
            // 合并整个路径
            vector<int> tmp_shortest_path = sequence;
            int _last = shortest_path[sequence[sequence.size()-1]][MAX_NODE_NUM-1];
            vector<int> tmp_path;
            tmp_path.push_back(49);
            while (_last != sequence[sequence.size()-1]) {
                tmp_path.push_back(_last);
                _last = shortest_path[sequence[sequence.size()-1]][_last];
            }
            reverse(tmp_path.begin(), tmp_path.end());
            tmp_shortest_path.insert(tmp_shortest_path.end(), tmp_path.begin(), tmp_path.end());
            // 计算该路径的费用代价
            int total_cost = 0;
            for (int i = 1; i < tmp_shortest_path.size(); i++) {
                total_cost += cost_matrix[tmp_shortest_path[i-1]][tmp_shortest_path[i]];
            }
            // 若费用满足要求,则更新最优解
            if (total_cost <= COST_UPPER_BOUND) {
                current_shortest_dist = opt_distance;
                current_shortest_path = tmp_shortest_path;
            }
        }
        // 继续前进
        // 二叉树
        // 找到不在sequence和remove中且可以与last_node连通的结点c
        for (int i = 0; i < MAX_NODE_NUM; i++){
            vector<int>::iterator it_seq;
            vector<int>::iterator it_rm;
            it_seq = find(sequence.begin(), sequence.end(), i);
            it_rm = find(remove.begin(), remove.end(), i);
            if (i != last_node && it_seq == sequence.end() && it_rm == remove.end() &&
                distance_matrix[last_node][i] < INT_MAX_VALUE) {
                // 左子树
                sequence.push_back(i);
                vector<int> tmp_remove;
                search(sequence, tmp_remove, sequence_distance+distance_matrix[sequence[sequence.size()-2]][sequence[sequence.size()-1]], sequence_cost+cost_matrix[sequence[sequence.size()-2]][sequence[sequence.size()-1]], current_shortest_dist, current_shortest_path, shortest_path, shortest_distance, min_cost, distance_matrix, cost_matrix);
                // 右子树
                sequence.pop_back();
                remove.push_back(i);
                search(sequence, remove, sequence_distance, sequence_cost, current_shortest_dist, current_shortest_path, shortest_path, shortest_distance, min_cost, distance_matrix, cost_matrix);
                break;
            }
        }
    }
}

int main()
{
    int distance_matrix[MAX_NODE_NUM][MAX_NODE_NUM]; // 距离矩阵
    int cost_matrix[MAX_NODE_NUM][MAX_NODE_NUM]; // 费用矩阵
    int shortest_distance[MAX_NODE_NUM-1][MAX_NODE_NUM]; // 记录各节点间的最短路径长度
    int min_cost[MAX_NODE_NUM-1][MAX_NODE_NUM]; // 记录各节点间的最小费用
    int shortest_path[MAX_NODE_NUM-1][MAX_NODE_NUM]; // 记录各节点间的最短路径
    int min_cost_path[MAX_NODE_NUM-1][MAX_NODE_NUM]; // 记录各节点间的最小费用对应的路径

    memset(shortest_distance, 0, sizeof(int)*MAX_NODE_NUM*MAX_NODE_NUM);

    char distance_file[] = "m1.txt";
    char cost_file[] = "m2.txt";
    parse_input(distance_file, distance_matrix);
    parse_input(cost_file, cost_matrix);
    TI(time);
    // 用dijkstra算法找出在没有其余条件限制下,各节点到B的最短路径最小费用
    for (int i = 0; i < MAX_NODE_NUM - 1; i++) {
        dijkstra(distance_matrix, i, shortest_distance[i], shortest_path[i]);
        dijkstra(cost_matrix, i, min_cost[i], min_cost_path[i]);
    }

    // 分支定界法求有过路费约束的从A到B的最短路径
    vector<int> sequence;
    vector<int> remove;
    vector<int> current_shortest_path;
    sequence.push_back(0);
    int current_shortest_dist = INT_MAX_VALUE;
    search(sequence, remove, 0, 0, current_shortest_dist, current_shortest_path, shortest_path, shortest_distance, min_cost, distance_matrix, cost_matrix);
    TO(time, "Total");
    print_result(current_shortest_dist, current_shortest_path, cost_matrix);

    return 0;
}
\end{lstlisting}
	\subsubsection{多叉搜索树}
\begin{lstlisting}
/*
 * solve single source shortest path problem using branch and bound algorithm
 */

#include <iostream>
#include <fstream>
#include <string.h>
#include <vector>
#include <algorithm>
#include "timer.h"

#define MAX_NODE_NUM 50
#define INT_MAX_VALUE 9999
#define COST_UPPER_BOUND 1500

using std::ifstream;
using std::vector;
using std::find;
using std::cout;
using std::endl;
using std::reverse;

class Path
{
    public:
        int distance;
        int cost;
        vector<int> nodes;

        Path();
        void add_node(int node, int distance_matrix[][MAX_NODE_NUM], int cost_matrix[][MAX_NODE_NUM]);
        void set_path(int source, int shortest_distance[], int shortest_path[]);
        void update_cost(int cost_matrix[][MAX_NODE_NUM]);
        int length();
        int last();
        void add_path(Path path, int start);
        bool has(int node);
        void pop_back(int distance_matrix[][MAX_NODE_NUM], int cost_matrix[][MAX_NODE_NUM]);
};

Path::Path()
{
    this->distance = 0;
    this->cost = 0;
}

void Path::add_node(int node, int distance_matrix[][MAX_NODE_NUM], int cost_matrix[][MAX_NODE_NUM])
{
    this->nodes.push_back(node);
    int node_num = this->nodes.size();
    if (node_num >= 2) {
        int start = this->nodes[node_num-2];
        int end = this->nodes[node_num-1];
        this->distance += distance_matrix[start][end];
        this->cost += cost_matrix[start][end];
    }
}

void Path::set_path(int source, int shortest_distance[], int shortest_path[])
{
    int i = MAX_NODE_NUM-1;
    do {
        this->nodes.push_back(i);
        i = shortest_path[i];
    } while (i != source);
    if (source != MAX_NODE_NUM-1)
        this->nodes.push_back(i);
    reverse(this->nodes.begin(), this->nodes.end());
    this->distance = shortest_distance[MAX_NODE_NUM-1];
}

void Path::update_cost(int cost_matrix[][MAX_NODE_NUM])
{
    this->cost = 0;
    for (int i = 1; i < this->nodes.size(); i++) {
        this->cost += cost_matrix[this->nodes[i-1]][this->nodes[i]];
    }
}

int Path::length()
{
    return this->nodes.size();
}

int Path::last()
{
    return this->nodes[this->nodes.size()-1];
}

void Path::add_path(Path path, int start)
{
    this->nodes.insert(this->nodes.end(), path.nodes.begin()+start, path.nodes.end());
    this->distance += path.distance;
    this->cost += path.cost;
}

bool Path::has(int node)
{
    bool flag = false;
    for (int i = 0; i < this->nodes.size(); i++) {
        if (node == this->nodes[i]) {
            flag = true;
            break;
        }
    }
    return flag;
}

void Path::pop_back(int distance_matrix[][MAX_NODE_NUM], int cost_matrix[][MAX_NODE_NUM])
{
    int last = this->nodes.size();
    int start = this->nodes[last-2];
    int end = this->nodes[last-1];
    this->distance -= distance_matrix[start][end];
    this->cost -= cost_matrix[start][end];
    this->nodes.pop_back();
}

void parse_input(char *filename, int distance_matrix[][MAX_NODE_NUM])
{
    ifstream f(filename);
    for (int i = 0; i < MAX_NODE_NUM; i++) {
        for (int j = 0; j < MAX_NODE_NUM; j++) {
            f >> distance_matrix[i][j];
        }
    }
}

void print_path(Path path)
{
    if (path.length() == 0) {
        cout << "No Solution" << endl;
        return;
    }
    cout << "path: ";
    for (int i = 0; i < path.length()-1; i++) {
        cout << path.nodes[i]+1 << "->";
    }
    cout << path.last()+1 << endl;
    cout << "distance: " << path.distance << endl;
    cout << "cost: " << path.cost << endl;
}

void dijkstra(int distance_matrix[][MAX_NODE_NUM], int source, Path &path)
{
    // found用于标记已经得到最短距离的目标结点
    int shortest_distance[MAX_NODE_NUM];
    int shortest_path[MAX_NODE_NUM];
    static int found[MAX_NODE_NUM];
    // 初始化为1
    memset(found, 0, sizeof(found));

    // 找出从源点到其相邻节点的最短路径
    found[source] = 1;
    for (int i = 0; i < MAX_NODE_NUM; i++) {
        shortest_distance[i] = distance_matrix[source][i];
        shortest_path[i] = source;
    }

    int min_distance;
    int min_distance_idx;

    // 找到从源点出发到其他各结点的最短路径
    for (int i = 1; i < MAX_NODE_NUM; i++) {
        min_distance = INT_MAX_VALUE;
        // 找到未被标记的路径最短的结点
        for (int j = 0; j < MAX_NODE_NUM; j++) {
            if (found[j] == 0 && shortest_distance[j] < min_distance) {
                min_distance = shortest_distance[j];
                min_distance_idx = j;
            }
        }
        // 标记该结点为找到最短路径
        found[min_distance_idx] = 1;

        // 利用该该结点的最短路径更新其他结点的最短路径
        for (int j = 0; j < MAX_NODE_NUM; j++) {
            int dist_idx_j = distance_matrix[min_distance_idx][j]; // 从该结点到j的路径
            int new_distance = min_distance + dist_idx_j; // 从source到该结点的最短路径+从该结点到j的路径
            if (found[j] == 0 && dist_idx_j != INT_MAX_VALUE && new_distance < shortest_distance[j]) {
                shortest_distance[j] = new_distance;
                shortest_path[j] = min_distance_idx;
            }
        }
    }
    path.set_path(source, shortest_distance, shortest_path);

}

void search(Path sequence, Path &solution,
            Path shortest_path[], Path min_cost_path[],
            int distance_matrix[][MAX_NODE_NUM], int cost_matrix[][MAX_NODE_NUM])
{
    int last_node = sequence.last();
    int opt_distance = sequence.distance + shortest_path[last_node].distance;
    int opt_cost = sequence.cost + min_cost_path[last_node].cost;
    // 符合剪枝条件,则回溯
    //   1. sequence长度 + sequence末尾节点到终点的最短路径长度 > 当前最优解
    //   2. sequence费用 + min_cost[sequence[-1]][MAX_NODE_NUM-1] > 1500
    //   剪枝条件: 0 且 (1或2)
    if (((shortest_path[last_node].distance > 0 &&
        opt_distance >= solution.distance) || opt_cost > COST_UPPER_BOUND)) {
        // 剪枝回溯
        return;
    } else {
        // 判断是否是可行解
        if (shortest_path[last_node].distance > 0) {
            // 当前序列去往B的后继路径存在时,才可能产生可行解
            // 合并整个路径
            Path tmp;
            tmp.add_path(sequence, 0);
            tmp.add_path(shortest_path[last_node], 1);
            // 若费用满足要求,则更新最优解
            if (tmp.cost <= COST_UPPER_BOUND) {
                solution = tmp;
            }
        }
        // 继续前进
        // 找到不在sequence和且可以与last_node连通的结点c
        for (int i = 0; i < MAX_NODE_NUM; i++){
            if (!sequence.has(i) &&
                distance_matrix[last_node][i] < INT_MAX_VALUE) {
                // 左子树
                sequence.add_node(i, distance_matrix, cost_matrix);
                search(sequence, solution, shortest_path, min_cost_path, distance_matrix, cost_matrix);
                sequence.pop_back(distance_matrix, cost_matrix);
            }
        }
    }
}

int main()
{
    int distance_matrix[MAX_NODE_NUM][MAX_NODE_NUM]; // 距离矩阵
    int cost_matrix[MAX_NODE_NUM][MAX_NODE_NUM]; // 费用矩阵
    Path shortest_path[MAX_NODE_NUM-1]; // 记录各节点间的最短路径
    Path min_cost_path[MAX_NODE_NUM-1]; // 记录各节点间的最小费用对应的路径

    char distance_file[] = "m1.txt";
    char cost_file[] = "m2.txt";
    parse_input(distance_file, distance_matrix);
    parse_input(cost_file, cost_matrix);
    TI(time);
    // 用dijkstra算法找出在没有其余条件限制下,各节点到B的最短路径和最小费用
    for (int i = 0; i < MAX_NODE_NUM - 1; i++) {
        dijkstra(distance_matrix, i, shortest_path[i]);
        // 更新最短路径的费用
        shortest_path[i].update_cost(cost_matrix);
        dijkstra(cost_matrix, i, min_cost_path[i]);
    }

    // 分支定界法求有过路费约束的从A到B的最短路径
    Path sequence;
    Path solution;
    solution.distance = INT_MAX_VALUE;
    sequence.add_node(0, distance_matrix, cost_matrix);
    int current_shortest_dist = INT_MAX_VALUE;
    search(sequence, solution, shortest_path, min_cost_path, distance_matrix, cost_matrix);
    TO(time, "Total");
    print_path(solution);

    return 0;
}
\end{lstlisting}
	\subsection{实验结果和耗时}
	两程序得到相同结果:最短路径为 1->3->8->11->15->21->23->26->32->37->39->45->47->50 \\
    对应的最短距离为464,总的费用为1448。	
    分别运行两程序1000次,计算平均时间得:二叉搜索树版本耗时3.044ms,多叉搜索树版本耗时2.487ms。
    \subsection{复杂度分析}
    由于dijkstra算法的复杂度为$O(n^2)$,算法对B之外的n-1个城市调用了dijkstra算法计算其到B的最短路径,所以这部分复杂度为$O(n^3)$,而后面递归搜索的过程复杂度与所有城市之间的边数有关,由于使用了定界限制,减少了分支数目,使复杂度大大降低。而多叉树在搜索时不需进行城市的排除,所以比二叉树搜索策略更快一些。

\end{document}